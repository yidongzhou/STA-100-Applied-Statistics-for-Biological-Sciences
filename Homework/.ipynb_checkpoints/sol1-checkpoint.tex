\documentclass{article}
\usepackage{graphicx,amsmath,amsthm} % Required for inserting images
\usepackage[margin=1in]{geometry}

\title{STA 100 Homework 1 Solution}
\author{Yidong Zhou}
\date{}

\begin{document}

\maketitle
\begin{enumerate}
    \item Please find below answers and justifications.
    \begin{enumerate}
        \item Observational. This study should be observational because the researcher is observing and analyzing data that already exist without any manipulation or intervention. It is not feasible or ethical to assign individuals to smoking or non-smoking groups.
        \item Experimental. This study should be experimental because the researcher is actively manipulating the treatment assignment by randomly assigning participants to different groups and then comparing the outcomes between the groups.
        \item Observational. This study should be observational because the researcher is observing and measuring variables without any intervention or manipulation of the participants' exercise habits.
		\item Experimental. This study should be experimental because the researcher is actively manipulating the teaching method by assigning students to different classes and assessing the impact of the new method on academic performance.
		\item Observational. This study should be observational because the researcher is observing and comparing individuals in different environmental conditions, without any active manipulation or intervention on their exposure to pollution.
    \end{enumerate}
    \item These variables are
    \begin{enumerate}
        \item discrete
        \item continuous
        \item nominal
        \item ordinal
    \end{enumerate}
    \item 
    \begin{enumerate}
        \item The average number of times a student exercised is
    $$\bar{X}=\frac{1}{n}\sum_{i=1}^nX_i=\frac{1}{100}(0\times20+1\times40+2\times24+3\times14+10\times2)=1.5.$$
        \item The median of the number of times a student exercised is the average of the $50$th and $51$th sorted observations, both of which are equal to $1$. The median is thus $(1+1)/2=1$.
        \item The variance of the number of times a student exercised is
    \begin{align*}
        s^2&=\frac{1}{n-1}\sum(X_i-\bar{X})^2\\&=\frac{1}{100-1}[20\times(0-1.5)^2+40\times(1-1.5)^2+24\times(2-1.5)^2+14\times(3-1.5)^2+2\times(10-1.5)^2]\\&=2.39.
    \end{align*}
        \item The standard deviation of the number of times a student exercised is $s=\sqrt{s^2}=\sqrt{2.39}=1.55$.
        \item The first quartile of the number of times a student exercised is the average of the $25$th and $26$th sorted observations, both of which are equal to $1$. The first quartile is thus $Q_1=(1+1)/2=1$.
        \item The third quartile of the number of times a student exercised is the average of the $75$th and $76$th sorted observations, both of which are equal to $2$. The third quartile is thus $Q_3=(2+2)/2=2$.
        \item Interquartile range is $\mathrm{IQR}=Q_3-Q_1=2-1=1$, so the lower fence is $\mathrm{LF}=Q_1-1.5\mathrm{IQR}=1-1.5\times1=-0.5$.
        \item The upper fence is $\mathrm{UF}=Q_3+1.5\mathrm{IQR}=2+1.5\times1=3.5$.
        \item The outliers in the data set are the two $10$'s since they are greater than the upper fence.
    \end{enumerate}
    \item 
    \begin{enumerate}
        \item FALSE. The standard deviation measures the spread of the dataset, whereas the mean measures the center. We could have a dataset that has a very small spread but a large center. For example, a dataset consists of ten values of $1000$ has a mean of $1000$ and a standard deviation of $0$.
        \item FALSE. The range of a dataset is defined as the difference between the maximum and minimum values of the dataset, and both the maximum and minimum could potentially be outliers (extreme points outside the main bulk of the data). So having extreme points outside the main bulk of the data versus not having those extreme points will have a strong influence on the range of the dataset.
		\item TRUE. By definition, the $q$th percentile of a dataset is the value for which $q\%$ of the data is below it and $(100-q)\%$ is above. Here $q=90$.
		\item TRUE. Since the mean of a dataset is the average of all the values in the dataset, of course including its outliers, the values of those extreme points have a strong influence on the mean.
    \end{enumerate}
    \item Let $S_1=$ Species 1, $S_2=$ Species 2, and $I=$ Infected.
    \begin{enumerate}
        \item $P(S_1)=\frac{54}{109}=0.4954.$
        \item $P(I)=\frac{58}{109}=0.5321.$
        \item $P(S_1\cap I)=\frac{38}{109}=0.3486.$
        \item $P(S_2\cap I^C)=\frac{35}{109}=0.3211.$
    \end{enumerate}
    \item 
    \begin{enumerate}
        \item $P(I|S_1)=\frac{P(S_1\cap I)}{P(S_1)}=\frac{38/109}{54/109}=0.7037.$
        \item $P(I|S_2)=\frac{P(S_2\cap I)}{P(S_2)}=\frac{20/109}{55/109}=0.3636.$
        \item $P(S_1|I)=\frac{P(S_1\cap I)}{P(I)}=\frac{38/109}{58/109}=0.6552.$
        \item $P(S_2|I)=\frac{P(S_2\cap I)}{P(I)}=\frac{20/109}{58/109}=0.3448.$
        \item Since $P(I|S_1)=0.7037\neq P(I)=0.5321$, they are not independent.
    \end{enumerate}
    \item Let $+$ and $-$ denote the events testing positive and negative, respectively. Define $D$ as the event that someone has the disease and $D^C$ its complement (someone does not have the disease).
    \begin{enumerate}
        \item $P(+\cap D)=P(+|D)P(D)=0.95\times0.04=0.038.$
        \item 
        \begin{align*}
        P(+)&=P(+\cap D)+P(+\cap D^C)\\&=P(+|D)P(D)+P(+|D^C)P(D^C)\\&=0.95\times0.04+(1-0.99)\times(1-0.04)\\&=0.0476.
      	\end{align*}
      	\item $$P(D|+)=\frac{P(+\cap D)}{P(+)}=\frac{0.038}{0.0476}=0.7983.$$
      	\item 
      	\begin{align*}
        P(D^C|-)&=\frac{P(-\cap D^C)}{P(-)}\\&=\frac{P(-|D^C)P(D^C)}{P(-|D^C)P(D^C)+P(-|D)P(D)}\\&=\frac{0.99\times(1-0.04)}{0.99\times(1-0.04)+(1-0.95)\times0.04}\\&=0.9979.
      	\end{align*}
      	\begin{align*}
        P(D^C|-)&=\frac{P(-\cap D^C)}{P(-)}\\&=\frac{P(-|D^C)P(D^C)}{P(-|D^C)P(D^C)+P(-|D)P(D)}\\&=\frac{0.99\times(1-0.02)}{0.99\times(1-0.02)+(1-0.9)\times0.04}\\&=0.9959.
      	\end{align*}
    \end{enumerate}
    \item 
    \begin{enumerate}
        \item The number of the hairs Yidong can expect to lose on a given morning is the expectation of $X$.
    	\begin{align*}
        E(X)&=\sum_{i=1}^5i\times P(X=i)\\&=1\times0.05+2\times0.1+3\times0.5+4\times0.3+5\times0.05\\&=3.2.
    	\end{align*}
    	\begin{align*}
        \mathrm{Var}(X)&=\sum_{i=1}^5(i-E(X))^2\times P(X=i)\\&=(1-3 .2)^2\times0.05+(2-3.2)^2\times0.1+(3-3.2)^2\times0.5+(4-3.2)^2\times0.3+(5-3.2)^2\times0.05\\&=0.76.
    	\end{align*}
    	The standard deviation of $X$ is thus $\sqrt{\mathrm{Var}(X)}=\sqrt{0.76}=0.87$.
    	\item Let $A$ denote the event that Yidong loses at $2$ hairs on a given morning. Let $B$ denote the event that Yidong loses at least $2$ hairs per day over $10$ days. If follows that
    	$$P(A)=1-P(A^C)=1-P(X=1)=1-0.05=0.95.$$
    	Note that each of the days are independent. One has
    	$$P(B)=P(A)^{10}=0.95^{10}=0.60.$$
    	\item Let $Y$ denote the number of the days that Yidong loses $5$ hairs over $10$ days. Then $Y$ is binomial distributed with parameters $n=10$ and $p=P(X=5)=0.05$. The probability that Yidong loses $5$ hairs at least one day over $10$ days is 
    $$P(Y\geq1)=1-P(Y=0)=1-0.95^{10}=0.40.$$
    \end{enumerate}
    \item Here $X\sim B(12, 0.37)$.
    \begin{enumerate}
        \item $$P(X=0)=\binom{12}{0}0.37^0(1-0.37)^{12}=0.04.$$
        \item \begin{align*}
        P(X\geq11)&=P(X=11)+P(X=12)\\&=\binom{12}{11}0.37^{11}(1-0.37)^1+\binom{12}{12}0.37^{12}(1-0.37)^0\\&=0.00014.
      	\end{align*}
      	\item \begin{align*}
        P(4<X\leq6)&=P(X=5)+P(X=6)\\&=\binom{12}{5}0.37^5(1-0.37)^7+\binom{12}{6}0.37^6(1-0.37)^6\\&=0.365.
      	\end{align*}
      	\item $$E(X)=np=12\times0.37=4.44.$$
    $$\mathrm{Var}(X)=np(1-p)=12\times0.37\times(1-0.37)=2.7972.$$
    \end{enumerate}
\end{enumerate}


\end{document}
